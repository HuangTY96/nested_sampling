%
% svn info 
% $Date$
% $Rev$
%
\documentclass[a4paper]{article}

%\usepackage[numbers]{natbib}
\usepackage[top=2.5cm,bottom=2.5cm]{geometry}
\usepackage{graphicx}
\usepackage{hyperref}
%\usepackage{amsmath}
\bibliographystyle{naturemagurl}
%\bibliographystyle{plainnat}


\title{Notes on coupling nested sampling with basin hopping}
\author{Jacob Stevenson and Stefano Martiniani}
\date{\today}

\begin{document}
\maketitle


\section{Nested Sampling}

The nested sampling algorithm is:
\begin{enumerate}
  \item Start by initializing $N_r$ replicas.  These should be completely random.  Some will have extremely high 
    temperatures, but that is fine.

  \item Remove the replica with the highest energy from your list and save the energy as $E_{max}$

  \item Get a configuration, $x$, sampled uniformly in configuration space
    with the only criterion that $E(x) \le E_{max}$.  Note that this is not trivial and will
    be discussed further below
  
  \item Add $x$ to the collection of replicas

  \item goto 2.

  \item the density of states and all other thermodynamic parameters can be calculated easily from the list of energies $E_{max}$

\end{enumerate}


The second step, ``sample $x$ uniformly in configuration space such that $E(x) < E_{max}$'' is
impossible to do exactly, so we explore phase space iterative for a given
number of steps and let $x$ be the stopping configuration.
It is generally done as follows:

\begin{enumerate}
  \item select a replica $x_r$ randomly from our collection of $N_r - 1$ replicas
  
  \item use $x_r$ as a starting point for a monte carlo chain.  The acceptance condition 
    of the monte carlo chain is simply: accept if $E_{new} < E_{max}$.  
\end{enumerate}

Note that this is not thermodynamic sampling.  We want to sample in an unbiased
way from all of configuration space with $E(x) < E_{max}$


\section{Sampling from the harmonic superposition approximation}

A weakness of the nested sampling algorithm is one of sampling.  Imagine the following situation
of trying to describe a crystalization transition.  At high energies the liquid state
completely dominates phase spase.  Nearly all of the replicas will be in the disordered liquid
configurations.  As $E_{max}$ decreases the relative weight of the crystal phse becomes more 
and more important, and you would start to expect to see some replicas in the crystal.  This is
your standard sampling problem that plagues molecular dynamics and monte carlo.
It can be really really difficult to find the crystal.  If you end up in a
situation where no replicas are in the crystal phase your results will then
completely miss the transition.  It is even worse for nested sampling because $E_{max}$ decreases
monotonically. For $E_{max}$ very high it's unlikely to be in the crystal because the phase space 
volume is so small.  For $E_{max}$ very low the relative phase volume is very
favorable, but exploring the energy landscape is very restricted so it is very
unlikely, or even impossible to find it starting from a disordered
configutation.  There is a relatively small range of $E_{max}$ where enough stars align so that
it is likely that the crystal is found.

We will use Basinhopping to cacluate the ground state and all relevant low lying minima.
We will then use this information to improve the step of the nested sampling
algorithm ``sample uniformly from configuration space with $E(x) < E_{max}$''.

In the harmonic approximation the density of states of minimum $\alpha$ is

\begin{equation}
  \Omega(E) = \frac{2N!(E - E_\alpha)^{k-1} }{\Gamma(k) \bar{\nu}^k O_\alpha}
\end{equation}

$E_\alpha$ is the energy of the minimum.  $\Gamma(k)$ is the gamma function $(k-1)!$, $k$ is the number
of vibrational degrees of freedom ($3*N-3$ for a cluster).  $\bar{\nu}$ is the geometric mean of the positive normal mode frequencies, or, put another way, $\bar{\nu}^k$ is the product of the positive normal mode frequencies $\nu_{\alpha}$.  $2N!$ accounts for all the nonsuperimposible permutation-inversion isomers of each structure. $O_\alpha$ is the order of the symmetry point group of the structure of minimum $\alpha$.  The $2N!$ permutations and the gamma function are the same for all structures and will cancel out when renormalising.
To get the phase space volume of the basin of a minimum we integrate the energy

\begin{equation}
  V_{\alpha}(E_{max}) = \int_{E_{\alpha}}^{E_{max}} \Omega(E) dE = \frac{2N!(E_{max} - E_\alpha)^{k} }{\Gamma(k+1) \bar{\nu}^k O_\alpha}
\end{equation}

To sample uniformly from the harmonic approximation we

\begin{enumerate}
  \item select a minimum $\alpha$ uniformly with weight $V_{\alpha} / V_{tot}$

  \item sample uniformly from with basin $\alpha$
\end{enumerate}

\subsection{Sample uniformly in a basin}

There are several methods of doing this.  We can use $x_{\alpha}$ as a starting
point for our monte carlo chain as described above.  We can displace using the known
eigenvalues and eigenvectors from the minimum.  Or we can do a combination.  Lets
first talk about how to use the eigenvalues $\mu_i$ and orthonormal eigenvectors $\vec{v_i}$ 

\begin{equation}
  \vec{x} = \vec{x}_{\alpha} + \sum_{i=1}^k q_i (\vec{v_i} / \sqrt{\mu_i})
  \label{eqn:step_away}
\end{equation}

We scale the vectors by $\vec{v_i} / \sqrt{\mu_i}$ so that all directions have
the same curvature.  The problem is now reduced to how to choose the $q_i$ such
that we are sampling uniformly in configuration space.  Scaling the vectors
above make all directions uniform, so the unit vector $\vec{q} / ||\vec{q}||$ should be 
uniform. (Note: sampling a unit vector uniformly is done by choosing all components from
a normal distribution and normalizing the resulting vector)

We are now left with determining the appropriate magnitude of the vector $||\vec{q}|| = q$.  Stepping in any one direction with a stepsize of $q$ 

\begin{equation}
  x = x_{\alpha} + q (\vec{v_i} / \sqrt{\mu_i})
\end{equation}
results in an energy change of
\begin{equation}
  E = E_{\alpha} + \frac{1}{2} q^2
  \label{eqn:echange}
\end{equation}

The total magnitude of the step defined in \ref{eqn:step_away} is $q =
\left( \sum{q_i^2} \right)^{\frac{1}{2}} = \left|\left|\vec{q}\right|\right|$ which results in a total energy change given, in
exactly the same way as \ref{eqn:echange}.  So $q$ can be at maximum

\begin{equation}
  q_{max} = \sqrt{2 (E_{max} - E_{\alpha})}
\end{equation}


But from what distribution should $q$ be drawn?  
We know that the density of states grows like $(E-E_{\alpha})^{k-1}$.
Thus the probability of drawing a configuration with energy
between $E$ and $E+dE$ should be 
\begin{equation}
  P(E)dE \propto (E-E_{\alpha})^{k-2}dE
\end{equation}
In $q$ space this probility density function becomes $P(q)dq = P(E)dE$
\begin{equation}
  P(q)dq \propto q^{2k-3}dq
\end{equation}

Thus we draw $q$ from a power law distribution with power $(2k-2)$ and maximum
value $\sqrt{2 (E_{max} - E_{\alpha})}$.  Note that the probability density
function above with power $k-1$ has a cumulative distribution function with
power $k$, so these distributions are refered to as power law distributions
with power $k$.
Alternatively we can draw the target change in energy $dE_{target}$ from a power law
distribution with power $k-1$ and maximum value $(E_{max} - E_{\alpha})$
and set $q = \sqrt{2 dE_{target}}$.




\bibliography{jakes_biblio}

\end{document}
